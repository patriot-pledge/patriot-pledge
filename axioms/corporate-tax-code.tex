\section{Corporate Tax Code}\label{sec:corporate-tax-code}
A fair and functional tax system is the bedrock of a healthy and equitable society.
Corporations must be held accountable to pay their fair share of taxes, and legal loopholes that allow tax evasion must be decisively closed.
Implementing comprehensive tax reform will ensure that businesses contribute appropriately to the communities and economies they benefit from, fostering a more just and sustainable economic environment.

\subsection*{Closing Corporate Tax Loopholes}
All corporate tax loopholes must be closed to ensure that businesses cannot exploit the system to avoid paying taxes.
These loopholes, often used to minimize tax liabilities, undermine public trust and deprive governments of essential revenue needed to fund public services and infrastructure.
By closing these loopholes, we can ensure a level playing field where all businesses pay their fair share, contributing to the common good and supporting a more equitable society.

\subsection*{Local Jurisdiction Taxation}
Businesses must pay taxes in the local jurisdiction for all transactions.
This means that income generated within a specific state should be subject to that state's income tax laws.
By ensuring that taxes are paid where economic activity occurs, we promote fair competition and prevent companies from shifting profits to low-tax jurisdictions to avoid paying their dues.

\subsection*{State-Specific Deductions}
Deductions toward state income taxes may only be derived from expenses incurred within that state.
This provision ensures that businesses cannot artificially inflate their deductions by including out-of-state expenses, which distorts the true economic activity within the state.
By restricting deductions to in-state expenses, we create a more accurate and fair taxation system that reflects the genuine costs of doing business locally.

\subsection*{Federal Income Tax on Post-State-Tax Profit}
After all state income taxes are deducted, federal income tax should be applied to the remaining profit (post-state-tax profit).
This approach ensures that businesses pay their share of federal taxes based on the actual profit retained after fulfilling their state tax obligations.
It promotes a fair distribution of tax responsibilities between state and federal levels, ensuring that each level of government receives the revenue it needs to serve the public effectively.

\subsection*{Treatment of International Expenses}
International expenses will not apply toward any state income tax deductions.
This rule prevents businesses from reducing their state tax liabilities through costs incurred abroad, ensuring that state taxes are based solely on domestic economic activities.
However, international expenses will apply toward federal income tax deductions.
This allows for a balanced approach where businesses can account for global operational costs at the federal level, reflecting the broader scope of national economic activity.

\section*{Conclusion}
Reforming corporate taxation and closing legal loopholes is essential for creating a fair and just economic system.
By ensuring that businesses pay taxes in the local jurisdictions where they operate and restricting deductions to relevant expenses, we can foster a more transparent and equitable tax system.
Applying federal taxes to post-state-tax profits and appropriately managing international expenses further ensures that corporations contribute their fair share to the communities and economies they benefit from.
These reforms are crucial for supporting public services, infrastructure, and the overall well-being of society, promoting a more stable, equitable, and prosperous future for all.

