\section{Strengthening Voting Rights and Democracy}\label{sec:voting-rights}
A strong democracy requires fair and inclusive voting rights, ensuring every citizen's voice is heard.
Several key reforms can enhance our electoral system: ranked choice voting, term limits, making Election Day a national holiday, guaranteeing access to voting booths and mail-in ballots, making gerrymandering federally illegal, and overturning Citizens United.

\subsection{Ranked Choice Voting}\label{subsec:ranked-choice-voting}
Ranked choice voting (RCV) allows voters to rank candidates by preference rather than selecting just one.
If no candidate wins a majority of first-choice votes, the candidate with the fewest votes is eliminated, and those votes are redistributed based on the next preferences.
This process continues until a candidate achieves a majority.
RCV encourages candidates to appeal to a broader base of voters, reducing negative campaigning and fostering more civil discourse.
It also allows voters to express their true preferences without fear of wasting their vote, which can lead to more satisfactory electoral outcomes.
Implementing RCV for all national elections ensures winners reflect a broader consensus of the electorate, promoting a more democratic and representative government.

\subsection{State-Level Adoption of Ranked Choice Voting}\label{subsec:state-level-adoption-of-ranked-choice-voting}
Encouraging the adoption of RCV on a state-by-state basis can serve as a proving ground for this electoral reform.
States can act as laboratories of democracy, showcasing the benefits and feasibility of RCV through successful implementations.
By demonstrating its advantages at the state level, such as increased voter satisfaction and more representative results, momentum can build for national adoption.
This approach allows for testing and refining the process, addressing potential challenges on a smaller scale before implementing it nationwide.
Encouraging states to adopt RCV ensures a gradual and evidence-based transition to a more equitable voting system across the country.

\subsection{Term Limits}\label{subsec:term-limits}
Term limits prevent the entrenchment of political power and encourage fresh perspectives in government.
By limiting the number of terms an individual can serve in a particular office, we can reduce the potential for corruption and complacency.
Term limits promote accountability, as officials are less likely to become disconnected from their constituents.
They ensure that elected positions remain accessible to new candidates, fostering a dynamic and responsive political environment.
Implementing term limits can revitalize democracy by regularly introducing new ideas and energy into the political process, helping to better address the evolving needs and concerns of the populace.

\subsection{Election Day as a National Holiday}\label{subsec:election-day-as-a-national-holiday}
Making Election Day a national holiday is a crucial step toward increasing voter participation.
Many Americans face challenges in getting to the polls due to work, school, or other commitments.
By designating Election Day as a national holiday, these barriers are removed, making it easier for citizens to exercise their right to vote.
A national holiday emphasizes the importance of voting and encourages higher turnout, leading to more representative and legitimate election outcomes.
This change would also create a celebratory atmosphere around the democratic process, reinforcing the civic duty and privilege of voting.

\subsection{Access to Voting Booths and Mail-In Ballots}\label{subsec:access-to-voting-booths-and-mail-in-ballots}
Ensuring all Americans have access to voting booths and mail-in ballots is fundamental to a fair electoral system.
Barriers to voting, such as long wait times, inadequate polling locations, and restrictive voter ID laws, disproportionately affect marginalized communities.
By guaranteeing access to voting booths and providing universal mail-in ballots, we can ensure that every citizen has the opportunity to participate in the democratic process.
These measures promote inclusivity and equity, reinforcing the principle that every vote counts.
Enhancing accessibility to voting strengthens democracy by ensuring that electoral outcomes genuinely reflect the will of the people.

\subsection{Making Gerrymandering Federally Illegal}\label{subsec:making-gerrymandering-federally-illegal}
Gerrymandering, the manipulation of electoral district boundaries to favor specific political parties, undermines the fairness of elections.
Making gerrymandering federally illegal would ensure that district boundaries are drawn impartially, reflecting genuine community interests rather than political advantage.
Independent redistricting commissions, free from partisan influence, can help achieve fair and transparent districting processes.
These commissions can utilize nonpartisan criteria and public input to draw boundaries that promote competitive elections and accurate representation.
Eliminating gerrymandering enhances the integrity of the electoral process, ensuring that elected officials are truly accountable to the voters.

\subsection{Overturning Citizens United}\label{subsec:overturning-citizens-united}
The Citizens United decision allows unlimited corporate spending in elections, significantly increasing the influence of money in politics.
Overturning Citizens United would reduce the outsized impact of wealthy donors and special interest groups, creating a more level playing field for all candidates.
This reform would help ensure that political campaigns focus on issues and policies rather than fundraising and financial influence.
By limiting corporate and special interest spending, we can promote a government that truly represents the people, enhancing the democratic process.
Overturning Citizens United is crucial for restoring public trust in the electoral system and ensuring that elected officials are accountable to their constituents rather than to financial backers.

\subsection*{Conclusion}
Strengthening voting rights is essential for a fair and inclusive democracy.
Implementing ranked choice voting, establishing term limits, making Election Day a national holiday, guaranteeing access to voting booths and mail-in ballots, making gerrymandering illegal, and overturning Citizens United are critical reforms.
These measures create a more representative, equitable, and dynamic political system where every citizen's voice is valued.
Embracing these reforms is vital for fortifying our democracy and building a more just society.


%\section{Voting Rights}
%
%\subsection*{Access to Voting is an Inalienable Right}
%
%Test test
%
%\subsection*{Election Day is a National Holiday}
%
%
%\subsection*{Ranked-Choice Voting}
%
%
%\subsection*{Ban Gerrymandering}
%Gerrymandering sucks
%
%It really does